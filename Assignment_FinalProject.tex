\PassOptionsToPackage{unicode=true}{hyperref} % options for packages loaded elsewhere
\PassOptionsToPackage{hyphens}{url}
%
\documentclass[]{article}
\usepackage{lmodern}
\usepackage{amssymb,amsmath}
\usepackage{ifxetex,ifluatex}
\usepackage{fixltx2e} % provides \textsubscript
\ifnum 0\ifxetex 1\fi\ifluatex 1\fi=0 % if pdftex
  \usepackage[T1]{fontenc}
  \usepackage[utf8]{inputenc}
  \usepackage{textcomp} % provides euro and other symbols
\else % if luatex or xelatex
  \usepackage{unicode-math}
  \defaultfontfeatures{Ligatures=TeX,Scale=MatchLowercase}
\fi
% use upquote if available, for straight quotes in verbatim environments
\IfFileExists{upquote.sty}{\usepackage{upquote}}{}
% use microtype if available
\IfFileExists{microtype.sty}{%
\usepackage[]{microtype}
\UseMicrotypeSet[protrusion]{basicmath} % disable protrusion for tt fonts
}{}
\IfFileExists{parskip.sty}{%
\usepackage{parskip}
}{% else
\setlength{\parindent}{0pt}
\setlength{\parskip}{6pt plus 2pt minus 1pt}
}
\usepackage{hyperref}
\hypersetup{
            pdftitle={Assignment\_FinalProject},
            pdfauthor={Danielle Hatt},
            pdfborder={0 0 0},
            breaklinks=true}
\urlstyle{same}  % don't use monospace font for urls
\usepackage[margin=1in]{geometry}
\usepackage{graphicx,grffile}
\makeatletter
\def\maxwidth{\ifdim\Gin@nat@width>\linewidth\linewidth\else\Gin@nat@width\fi}
\def\maxheight{\ifdim\Gin@nat@height>\textheight\textheight\else\Gin@nat@height\fi}
\makeatother
% Scale images if necessary, so that they will not overflow the page
% margins by default, and it is still possible to overwrite the defaults
% using explicit options in \includegraphics[width, height, ...]{}
\setkeys{Gin}{width=\maxwidth,height=\maxheight,keepaspectratio}
\setlength{\emergencystretch}{3em}  % prevent overfull lines
\providecommand{\tightlist}{%
  \setlength{\itemsep}{0pt}\setlength{\parskip}{0pt}}
\setcounter{secnumdepth}{0}
% Redefines (sub)paragraphs to behave more like sections
\ifx\paragraph\undefined\else
\let\oldparagraph\paragraph
\renewcommand{\paragraph}[1]{\oldparagraph{#1}\mbox{}}
\fi
\ifx\subparagraph\undefined\else
\let\oldsubparagraph\subparagraph
\renewcommand{\subparagraph}[1]{\oldsubparagraph{#1}\mbox{}}
\fi

% set default figure placement to htbp
\makeatletter
\def\fps@figure{htbp}
\makeatother


\title{Assignment\_FinalProject}
\author{Danielle Hatt}
\date{3/27/2020}

\begin{document}
\maketitle

\hypertarget{introduction}{%
\section{Introduction}\label{introduction}}

Seagrasses ecosystems are highly productive and distributed globally
(Fourqurean et al.~2012). Karstic sediments known to be produced by
calcifying organisms (Hill et al.~2015; Ortegon-Aznar et al.~2017)
support healthy seagrass ecosystems in the tropics and subtropics
(Zieman et al.~1989). In tropical ecosystems, these seagrass beds are
responsible for creating a stable environment and contributing to the
protection of shorelines (van Tussenbroek \& Santos 2011). Intermingled
within these segrass beds, calcareous green macroalgae (CGA) such as
species of the Bryopsidales (Udotea, Rhipocephalus, Penicillus, and
Halimeda) and Dasycladales (Acetabularia, Cymopolia, and Neomeris) play
an important role as engineering species producing calcareous sediments
that facilitate the development of these large seagrass beds in
subtropical and tropical ecosystems (Hillis-Colinvaux 1980, Williams
1990, van Tussenbroek and van Dijk 2007). Most of the marine carbonate
found in tropical ecosystems are produced by calcareous algae
(Hillis-Colinvaux 1980; Bach 1979).

Florida Bay is a coastal subtropical lagoon that contains one of the
largest expanses of seagrass beds in the world and extends approximately
5500 km2, ranging from the Everglades to the Florida Keys (Fourqurean et
al.~1992). These seagrass beds are high in biodiversity and support many
crucial and economically important organisms (Zieman et al.~1989). They
make up approximately ten percent of the expanse of seagrass beds found
in Florida Bay (Zieman et al.~1989).

\hypertarget{methods}{%
\section{Methods}\label{methods}}

\hypertarget{site-information}{%
\subsubsection{Site Information}\label{site-information}}

The Long-term Ecological Monitoring Program was conducted in the Florida
Coastal Everglades (LTER:FCE). Samples of calcareous green algae
obtained for this study were collected from three sites representative
of a salinity gradient in Florida Bay: Sprigger Bank (24°91'N, 80°93'W),
Bob Allen (25°02'N, 80°68'W) and Duck Key (25°17'N, 80°48'W) (Figure 1).
Sprigger Bank is the only site where both \emph{Halimeda} and
\emph{Penicillus} were present. \emph{Penicillus} was observed at all
three sites. In terms of salinity, Sprigger Bank is the most stable site
and is dominated by marine waters while Bob Allen and Duck Key are
affected by freshwater from Everglades sources due to its close
proximity to land (Herbert \& Fourqurean 2009; Frankovich \& Fourqurean
2009).

Florida Bay is a shallow coastal subtropical lagoon that contains one of
the largest expanses of seagrass beds in the world extending
approximately 5500 km2, ranging from the Everglades to the Florida Keys
(Fourqurean et al.~1992). The dominant seagrasses in the bay are
Thalassia testudinum, Halodule wrightii and Syringodium filiforme with
intermixed rhizophitic macroalgae species of the genera Halimeda,
Penicillus, Udotea, Caulerpa, and other red algae such as species of the
genera Laurencia and Amphiroa (Frankovich and Fourqurean 1997), with
calcareous green algae (CGA) making up approximately ten percent of the
expanse of the seagrass beds (Zieman et al.~1989). The distribution of
macrophyte species has been related with patterns of salinity and
nutrient availability and higher salinity fluctuations, varying from the
northeast characterized by higher availability of nitrogen and higher
variability of salinity, while the southwest regions of the bay is
characterized by more stable salinity within ranges of marine
conditions, and higher availability of phosphorous (Ziemman 1989;
Herbert and Fourqurean 2009; Frankovich and Fourqurean 1997). Florida
Bay is known to provide refuge and functions as nursery for important
fisheries in the region (Herbert and Fourqurean 2009), therefore, it has
an important role as a life supporting system as well as supporting the
economy of South Florida (Biber and Irlandi 2005). Based on the
heterogeneity of the bay and the described gradients of nutrients and
salinity we selected three representative sites: Sprigger Bank (24°91'N,
80°93'W), localized in the central west region of the bay; and Bob Allen
(25°02'N, 80°68'W) and Duck Key (25°17'N, 80°48'W), both localized in
the northeast region of the bay (Figure 1). Water depth at all three
sites were below 2 meters (Herbert and Fourqurean 2009). Sprigger Bank
is impacted by the flow of water from the Gulf and characterized by high
density of seagrasses dominated by T. testudinum, stable salinity and
high phosphorous availability (Herbert and Fourquerean 2009, Zieman et
al.~1989). Bob Allen and Duck Key are a mix of flat subtidal basins and
shallow intertidal regions both impacted by the flow of freshwater
sources. These sites are characterized by limited abundance of H.
wrightii, T. testudinum and Penicillus and low tidal energy, variable
salinity due to their proximity to freshwater sources from the
Everglades and higher availability of nitrogen (Herbert and Fourqurean
2009, Fourqurean et al.~1992, Frankovich and Fourqurean 1997, Zieman et
al.~1989). Sprigger Bank is the only site where both Halimeda and
Penicillus are present, while Penicillus is observed at all three sites.

\textbf{CLIMATE}

\begin{figure}
\centering
\includegraphics[width=0.4\textwidth,height=\textheight]{/Users/daniellehatt/Desktop/BSC 6926/FlBay_map2.jpg}
\caption{\textbf{Figure 1 showing study sites at Florida Bay}}
\end{figure}

\hypertarget{methods-1}{%
\subsection{Methods}\label{methods-1}}

Surveys were conducted four times a year at the study site from 2007 to
2017. At each survey, divers used three randomly placed 0.25m\^{}2
quadrats along a transect line to collect macroalgae by hand. All
samples were brought back to the lab, cleaned and separated on the genus
level for each quadrat at each site. The samples were dried for 48 hours
in an oven set to 70oC. Samples for each quadrat were weighed and this
was recorded as the dry weight. The samples were ashed in an oven at
400oC for 5 hours. These ashes were weighed and were recorded as calcium
carbonate (CaCO3). CaCO3 was used as a proxy for inorganic carbon. The
weight of the CaCO3 recorded was subtracted from the dry weight
previously obtained and this new weight was used as the amount of
biomass for each quadrat. Biomass was used as a proxy for organic
carbon. These values were recorded as total mass (dry weight of the full
quadrat or individual thallus) and proportion of organic versus
inorganic carbon (of the full quadrat or individual thallus).

\hypertarget{statistical-analysis}{%
\subsection{Statistical Analysis}\label{statistical-analysis}}

The data for biomass of \emph{Penicillus} over the past 8 years was
cleaned to remove one significant spike. The data was decomposed using
multiplicative time series analysis to determine any underlying trends.
The data followed stationarity and autocorrelation assumptions. An Auto
Regressive Integrated Moving Average (ARIMA) model was fit and used to
forecast future values of biomass of \emph{Penicillus} in Florida Bay.
The residuals were plotted to determine its distribution and
significance. The AIC showed that the model fit the data.

The Augmented Dickey-Fuller Test will be used to test stationarity and
autocorrelation will be carried out on the data. If the data is not
stationary, differencing will be used to transform the data. When the
model is applied, residuals will be tested using autocorrelation and the
Ljung-Box test will be used to test for independence. The Akaike
Information Criteria (AIC) will be used to compare the model and
explatory variables.

\hypertarget{results}{%
\section{Results}\label{results}}

\begin{figure}
\centering
\includegraphics[width=0.6\textwidth,height=\textheight]{/Users/daniellehatt/Desktop/BSC 6926/Final Project Graphs/Rplot.jpeg}
\caption{\textbf{Figure 2 showing the four collections of biomass of
macroalgae from the site Sprigger Bank, Florida Bay from 2007 to 2017.}}
\end{figure}

\begin{figure}
\centering
\includegraphics[width=0.6\textwidth,height=\textheight]{/Users/daniellehatt/Desktop/BSC 6926/Final Project Graphs/Rplot01.jpeg}
\caption{\textbf{Figure 3 showing the \ldots{}.}}
\end{figure}

\begin{figure}
\centering
\includegraphics[width=0.6\textwidth,height=\textheight]{/Users/daniellehatt/Desktop/BSC 6926/Final Project Graphs/Rplot02.jpeg}
\caption{\textbf{Figure 3 showing the \ldots{}.}}
\end{figure}

\hypertarget{discussion}{%
\section{Discussion}\label{discussion}}

\hypertarget{references}{%
\section{References}\label{references}}

\end{document}
