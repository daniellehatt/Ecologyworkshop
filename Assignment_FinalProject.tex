\PassOptionsToPackage{unicode=true}{hyperref} % options for packages loaded elsewhere
\PassOptionsToPackage{hyphens}{url}
%
\documentclass[]{article}
\usepackage{lmodern}
\usepackage{amssymb,amsmath}
\usepackage{ifxetex,ifluatex}
\usepackage{fixltx2e} % provides \textsubscript
\ifnum 0\ifxetex 1\fi\ifluatex 1\fi=0 % if pdftex
  \usepackage[T1]{fontenc}
  \usepackage[utf8]{inputenc}
  \usepackage{textcomp} % provides euro and other symbols
\else % if luatex or xelatex
  \usepackage{unicode-math}
  \defaultfontfeatures{Ligatures=TeX,Scale=MatchLowercase}
\fi
% use upquote if available, for straight quotes in verbatim environments
\IfFileExists{upquote.sty}{\usepackage{upquote}}{}
% use microtype if available
\IfFileExists{microtype.sty}{%
\usepackage[]{microtype}
\UseMicrotypeSet[protrusion]{basicmath} % disable protrusion for tt fonts
}{}
\IfFileExists{parskip.sty}{%
\usepackage{parskip}
}{% else
\setlength{\parindent}{0pt}
\setlength{\parskip}{6pt plus 2pt minus 1pt}
}
\usepackage{hyperref}
\hypersetup{
            pdftitle={Assignment\_FinalProject},
            pdfauthor={Danielle Hatt},
            pdfborder={0 0 0},
            breaklinks=true}
\urlstyle{same}  % don't use monospace font for urls
\usepackage[margin=1in]{geometry}
\usepackage{graphicx,grffile}
\makeatletter
\def\maxwidth{\ifdim\Gin@nat@width>\linewidth\linewidth\else\Gin@nat@width\fi}
\def\maxheight{\ifdim\Gin@nat@height>\textheight\textheight\else\Gin@nat@height\fi}
\makeatother
% Scale images if necessary, so that they will not overflow the page
% margins by default, and it is still possible to overwrite the defaults
% using explicit options in \includegraphics[width, height, ...]{}
\setkeys{Gin}{width=\maxwidth,height=\maxheight,keepaspectratio}
\setlength{\emergencystretch}{3em}  % prevent overfull lines
\providecommand{\tightlist}{%
  \setlength{\itemsep}{0pt}\setlength{\parskip}{0pt}}
\setcounter{secnumdepth}{0}
% Redefines (sub)paragraphs to behave more like sections
\ifx\paragraph\undefined\else
\let\oldparagraph\paragraph
\renewcommand{\paragraph}[1]{\oldparagraph{#1}\mbox{}}
\fi
\ifx\subparagraph\undefined\else
\let\oldsubparagraph\subparagraph
\renewcommand{\subparagraph}[1]{\oldsubparagraph{#1}\mbox{}}
\fi

% set default figure placement to htbp
\makeatletter
\def\fps@figure{htbp}
\makeatother


\title{Assignment\_FinalProject}
\author{Danielle Hatt}
\date{3/27/2020}

\begin{document}
\maketitle

\hypertarget{introduction}{%
\section{Introduction}\label{introduction}}

Seagrasses ecosystems are highly productive and distributed globally
(Fourqurean et al. 2012). Karstic sediments known to be produced by
calcifying organisms (Hill et al. 2015; Ortegón-Aznar, Chuc-Contreras,
and Collado-Vides 2017) support healthy seagrass ecosystems in the
tropics and subtropics (Zieman, Fourqurean, and Iverson 1989). In
tropical ecosystems, these seagrass beds are responsible for creating a
stable environment and contributing to the protection of shorelines
(Tussenbroek and Barba Santos 2011). Intermingled within these seagrass
beds, calcareous green macroalgae (CGA) such as species of the
Bryopsidales (\emph{Udotea}, \emph{Rhipocephalus}, \emph{Penicillus},
and \emph{Halimeda}) and Dasycladales (\emph{Acetabularia},
\emph{Cymopolia}, and \emph{Neomeris}) play an important role as
engineering species producing calcareous sediments that facilitate the
development of these large seagrass beds in subtropical and tropical
ecosystems (Hillis-Colinvaux 1980; Tussenbroek and Dijk 2007). Most of
the marine carbonate found in tropical ecosystems are produced by
calcareous algae (Hillis-Colinvaux 1980; Bach 1979).

Florida Bay is a coastal subtropical lagoon that contains seagrass beds
high in biodiversity and support many crucial and economically important
organisms (Zieman, Fourqurean, and Iverson 1989). They make up
approximately ten percent of the expanse of seagrass beds found in
Florida Bay (Zieman, Fourqurean, and Iverson 1989). In Florida, the two
most abundant genera of calcareous green macroalgae are Halimeda and
Penicillus. The abundance of these genera fluctuates due to seasonal
variability; there is more growth and calcification recorded in summer
and autumn months from June to November when the sea surface
temperatures are above 20\textsuperscript{o}C (Wefer 1980;
Collado-Vides, Rutten, and Fourqurean 2005). It is particularly
important that the abundance of these communities be monitored due to
increased anthropogenic activities. In the Everglades region, there has
been a reduction of water flow due to the construction of canals, levees
and pumping stations to divert water and allow urbanization of South
Florida in the early 1900s. This resulted in a 70\% decrease in the
available water and continues to have devastating effects on surrounding
ecosystems. The Comprehensive Everglades Restoration Program (CERP) aims
to restore these historic water flow patterns over a thirty-year period
which may cause shifts in algal communities (Perry, n.d.).

The Florida Coastal Everglades, Long Term Ecological Research (FCE LTER)
program surveys calcareous algal communities at three sites
representative of a salinity gradient in Florida Bay: Sprigger Bank, Bob
Allen and Duck Key. In this polyhaline estuary, Sprigger Bank is more
stable in salinity compared to Bob Allen and Duck Key (Herbert and
Fourqurean 2009; Frankovich et al. 2009). Spatiotemporal long-term
studies like these helps to get a larger picture of changes occurring
over time, changes in slow biological processes or changing ecological
patterns that may not be evident otherwise (Franklin 1989). These
studies can also help to forecast potential trends in biomass as CERP
strategies and water management are being implemented at Florida Bay.
These changes can affect levels of salinity thereby causing reduced
production of organic and inorganic carbon between genera and among
different locations where these calcareous algal communities are present
in Florida Bay.

The hypotheses we propose are that (1)

\hypertarget{methods}{%
\section{Methods}\label{methods}}

\hypertarget{site-information}{%
\subsubsection{Site Information}\label{site-information}}

Florida Bay is a shallow coastal subtropical lagoon that contains one of
the largest expanses of seagrass beds in the world extending
approximately 5500 km2, ranging from the Everglades to the Florida Keys
(Fourqurean, Zieman, and Powell 1992). The dominant seagrasses in the
bay are \emph{Thalassia testudinum}, \emph{Halodule wrightii} and
\emph{Syringodium filiforme} with intermixed rhizophitic macroalgae
species of the genera \emph{Halimeda}, \emph{Penicillus}, \emph{Udotea},
\emph{Caulerpa}, and other red algae such as species of the genera
\emph{Laurencia} and \emph{Amphiroa} (Frankovich, Fourqureane, and Bay
1997), with calcareous green algae (CGA) making up approximately ten
percent of the expanse of the seagrass beds (Zieman, Fourqurean, and
Iverson 1989). The distribution of macrophyte species has been related
with patterns of salinity and nutrient availability and higher salinity
fluctuations, varying from the northeast characterized by higher
availability of nitrogen and higher variability of salinity, while the
southwest regions of the bay is characterized by more stable salinity
within ranges of marine conditions, and higher availability of
phosphorous (Zieman, Fourqurean, and Iverson 1989; Herbert and
Fourqurean 2009; Frankovich, Fourqureane, and Bay 1997).

Surveys of calcareous green algae were conducted at three sites
representative of a salinity gradient in Florida Bay: Sprigger Bank
(24°91'N, 80°93'W), Bob Allen (25°02'N, 80°68'W) and Duck Key (25°17'N,
80°48'W) (Figure 1). Sprigger Bank is the only site where both
\emph{Halimeda} and \emph{Penicillus} were present. \emph{Penicillus}
was observed at all three sites. Halimeda consists of segmented branches
and is considered a major tropical algae at shallow depths
(Collado-Vides, Rutten, and Fourqurean 2005). \emph{Penicillus} is
studied less frequently and has a different morphological structure
containing a calcified thallus and cap (Wefer 1980). Water depth at all
three sites were below 2 meters (Herbert and Fourqurean 2009). Sprigger
Bank is impacted by the flow of water from the Gulf and characterized by
high density of seagrasses dominated by T. testudinum, stable salinity
and high phosphorous availability (Zieman, Fourqurean, and Iverson 1989;
Herbert and Fourqurean 2009). Bob Allen and Duck Key are a mix of flat
subtidal basins and shallow intertidal regions both impacted by the flow
of freshwater sources. These sites are characterized by limited
abundance of \emph{H. wrightii}, \emph{T. testudinum} and
\emph{Penicillus} and low tidal energy, variable salinity due to their
proximity to freshwater sources from the Everglades and higher
availability of nitrogen (Zieman, Fourqurean, and Iverson 1989;
Fourqurean, Zieman, and Powell 1992; Frankovich, Fourqureane, and Bay
1997; Herbert and Fourqurean 2009).

\textbf{CLIMATE}

\begin{figure}
\centering
\includegraphics[width=0.4\textwidth,height=\textheight]{/Users/daniellehatt/Desktop/BSC 6926/FlBay_map2.jpg}
\caption{\textbf{Figure 1 showing study sites at Florida Bay}}
\end{figure}

\hypertarget{methods-1}{%
\subsection{Methods}\label{methods-1}}

Surveys were conducted four times a year at the study site from 2007 to
2017. At each survey, divers used three randomly placed
0.25m\textsuperscript{2} quadrats along a transect line to collect
macroalgae by hand. All samples were brought back to the lab, cleaned
and separated on the genus level for each quadrat at each site. The
samples were dried for 48 hours in an oven set to 70oC. Samples for each
quadrat were weighed and this was recorded as the dry weight. The
samples were ashed using the Loss on Ignition method (LOI) in an oven at
400oC for 5 hours (Fourqurean et al. 2012). These ashes were weighed and
were recorded as calcium carbonate (CaCO3). CaCO3 was used as a proxy
for inorganic carbon. The weight of the CaCO3 recorded was subtracted
from the dry weight previously obtained and this new weight was used as
the amount of biomass for each quadrat. Biomass was used as a proxy for
organic carbon.

\hypertarget{statistical-analysis}{%
\subsection{Statistical Analysis}\label{statistical-analysis}}

\hypertarget{results}{%
\section{Results}\label{results}}

\begin{figure}
\centering
\includegraphics[width=0.6\textwidth,height=\textheight]{/Users/daniellehatt/Desktop/BSC 6926/Final Project Graphs/Rplot.jpeg}
\caption{\textbf{Figure 2 showing the four collections of biomass of
macroalgae from the site Sprigger Bank, Florida Bay from 2007 to 2017.}}
\end{figure}

\begin{figure}
\centering
\includegraphics[width=0.6\textwidth,height=\textheight]{/Users/daniellehatt/Desktop/BSC 6926/Final Project Graphs/Rplot01.jpeg}
\caption{\textbf{Figure 3 showing the \ldots{}.}}
\end{figure}

\begin{figure}
\centering
\includegraphics[width=0.6\textwidth,height=\textheight]{/Users/daniellehatt/Desktop/BSC 6926/Final Project Graphs/Rplot03.jpeg}
\caption{\textbf{Figure 4 showing the \ldots{}.}}
\end{figure}

\includegraphics[width=0.6\textwidth,height=\textheight]{/Users/daniellehatt/Desktop/BSC 6926/Final Project Graphs/Rplot04.jpeg}

\includegraphics[width=0.6\textwidth,height=\textheight]{/Users/daniellehatt/Desktop/BSC 6926/Final Project Graphs/Rplot05.jpeg}

\begin{figure}
\centering
\includegraphics[width=0.6\textwidth,height=\textheight]{/Users/daniellehatt/Desktop/BSC 6926/Final Project Graphs/Rplot06.jpeg}
\caption{\textbf{Figure 7 showing the \ldots{}.}}
\end{figure}

\begin{figure}
\centering
\includegraphics[width=0.6\textwidth,height=\textheight]{/Users/daniellehatt/Desktop/BSC 6926/Final Project Graphs/Rplot07.jpeg}
\caption{\textbf{Figure 8 showing the \ldots{}.}}
\end{figure}

\begin{figure}
\centering
\includegraphics[width=0.6\textwidth,height=\textheight]{/Users/daniellehatt/Desktop/BSC 6926/Final Project Graphs/Rplot08.jpeg}
\caption{\textbf{Figure 9 showing the \ldots{}.}}
\end{figure}

\hypertarget{discussion}{%
\section{Discussion}\label{discussion}}

\#References

\setlength{\parindent}{-0.2in}
\setlength{\leftskip}{0.2in}
\setlength{\parskip}{8pt}

\noindent

\hypertarget{refs}{}
\leavevmode\hypertarget{ref-Bach1979}{}%
Bach, Steven D. 1979. ``STANDING CROP, GROWTH AND PRODUCTION OF
CALCAREOUS SIPHONALES (CHLOROPHYTA) IN A SOUTH FLORIDA LAGOON.'' 2. Vol.
29.
\href{https://www.ingentaconnect.com/content/umrsmas/bullmar/1979/00000029/00000002/art00005?crawler=true\%7B/\&\%7Dmimetype=application/pdf\%7B/\&\%7Dcasa\%7B/_\%7Dtoken=t5CU9TbnLX8AAAAA:BaQbRyYY\%7B/_\%7Dx7hfpwx3NTQnsQ6Jime\%7B/_\%7D5ShVKa2PLe7X3NAdf9tzmV3zYaliAN-SVXVILWUY\%7B/_\%7Dq7GvRSb1f13w}{https://www.ingentaconnect.com/content/umrsmas/bullmar/1979/00000029/00000002/art00005?crawler=true\{\textbackslash{}\&\}mimetype=application/pdf\{\textbackslash{}\&\}casa\{\textbackslash{}\_\}token=t5CU9TbnLX8AAAAA:BaQbRyYY\{\textbackslash{}\_\}x7hfpwx3NTQnsQ6Jime\{\textbackslash{}\_\}5ShVKa2PLe7X3NAdf9tzmV3zYaliAN-SVXVILWUY\{\textbackslash{}\_\}q7GvRSb1f13w}.

\leavevmode\hypertarget{ref-Collado-Vides2005}{}%
Collado-Vides, Ligia, Leanne M Rutten, and James W Fourqurean. 2005.
``SPATIOTEMPORAL VARIATION OF THE ABUNDANCE OF CALCAREOUS GREEN
MACROALGAE IN THE FLORIDA KEYS: A STUDY OF SYNCHRONY WITHIN A MACROALGAL
FUNCTIONAL-FORM GROUP 1.'' \emph{Journal of Phycology} 41: 742--52.
\url{https://doi.org/10.1111/j.1529-8817.2005.00099.x}.

\leavevmode\hypertarget{ref-Fourqurean2012}{}%
Fourqurean, James W., Carlos M. Duarte, Hilary Kennedy, Núria Marbà,
Marianne Holmer, Miguel Angel Mateo, Eugenia T. Apostolaki, et al. 2012.
``Seagrass ecosystems as a globally significant carbon stock.''
\emph{Nature Geoscience} 5 (7). Nature Publishing Group: 505--9.
\url{https://doi.org/10.1038/ngeo1477}.

\leavevmode\hypertarget{ref-Fourqurean1992}{}%
Fourqurean, James W., Joseph C. Zieman, and George V. N. Powell. 1992.
``Phosphorus limitation of primary production in Florida Bay: Evidence
from C:N:P ratios of the dominant seagrass Thalassia testudinum.''
\emph{Limnology and Oceanography} 37 (1). John Wiley \& Sons, Ltd:
162--71. \url{https://doi.org/10.4319/lo.1992.37.1.0162}.

\leavevmode\hypertarget{ref-Franklin1989}{}%
Franklin, Jerry F. 1989. ``Importance and Justification of Long-Term
Studies in Ecology.'' In \emph{Long-Term Studies in Ecology}, 3--19. New
York, NY: Springer New York.
\url{https://doi.org/10.1007/978-1-4615-7358-6_1}.

\leavevmode\hypertarget{ref-Frankovich2009}{}%
Frankovich, Thomas A., Anna R. Armitage, Ania H. Wachnicka, Evelyn E.
Gaiser, and James W. Fourqurean. 2009. ``NUTRIENT EFFECTS ON SEAGRASS
EPIPHYTE COMMUNITY STRUCTURE IN FLORIDA BAY.'' \emph{Journal of
Phycology} 45 (5). John Wiley \& Sons, Ltd: 1010--20.
\url{https://doi.org/10.1111/j.1529-8817.2009.00745.x}.

\leavevmode\hypertarget{ref-Frankovich1997}{}%
Frankovich, Thomas A, James W Fourqureane, and Florida Bay. 1997.
``Seagrass epiphyte loads along a nutrient availability gradient,
Florida Bay, USA.'' Vol. 159.
\url{https://www.int-res.com/articles/meps/159/m159p037.pdf}.

\leavevmode\hypertarget{ref-Herbert2009}{}%
Herbert, Darrell A., and James W. Fourqurean. 2009. ``Phosphorus
Availability and Salinity Control Productivity and Demography of the
Seagrass Thalassia testudinum in Florida Bay.'' \emph{Estuaries and
Coasts} 32 (1). Springer: 188--201.
\url{https://doi.org/10.1007/s12237-008-9116-x}.

\leavevmode\hypertarget{ref-Hill2015}{}%
Hill, Ross, Alecia Bellgrove, Peter I. Macreadie, Katherina Petrou, John
Beardall, Andy Steven, and Peter J. Ralph. 2015. ``Can macroalgae
contribute to blue carbon? An Australian perspective.'' \emph{Limnology
and Oceanography} 60 (5). John Wiley \& Sons, Ltd: 1689--1706.
\url{https://doi.org/10.1002/lno.10128}.

\leavevmode\hypertarget{ref-Hillis-Colinvaux1980}{}%
Hillis-Colinvaux, Llewellya. 1980. ``Ecology and Taxonomy of Halimeda:
Primary Producer of Coral Reefs.'' \emph{Advances in Marine Biology} 17
(January). Academic Press: 1--327.
\url{https://doi.org/10.1016/S0065-2881(08)60303-X}.

\leavevmode\hypertarget{ref-Ortegon-Aznar2017}{}%
Ortegón-Aznar, Ileana, Andrea Chuc-Contreras, and Ligia Collado-Vides.
2017. ``Calcareous green algae standing stock in a tropical sedimentary
coast.'' \emph{Journal of Applied Phycology} 29 (5). Springer
Netherlands: 2685--93. \url{https://doi.org/10.1007/s10811-017-1057-y}.

\leavevmode\hypertarget{ref-Perry}{}%
Perry, William. n.d. ``Elements of South Florida's Comprehensive
Everglades Restoration Plan.''
\url{www.saj.usace.army.mil/projects/index.html.}

\leavevmode\hypertarget{ref-VanTussenbroek2011}{}%
Tussenbroek, Brigitta I. van, and M. Guadalupe Barba Santos. 2011.
``Demography of Halimeda incrassata (Bryopsidales, Chlorophyta) in a
Caribbean reef lagoon.'' \emph{Marine Biology} 158 (7). Springer:
1461--71. \url{https://doi.org/10.1007/s00227-011-1662-2}.

\leavevmode\hypertarget{ref-VanTussenbroek2007}{}%
Tussenbroek, Brigitta I. van, and Jent Kornelis van Dijk. 2007.
``SPATIAL AND TEMPORAL VARIABILITY IN BIOMASS AND PRODUCTION OF
PSAMMOPHYTIC \textless{}i\textgreater{}HALIMEDA
INCRASSATA\textless{}/i\textgreater{} (BRYOPSIDALES, CHLOROPHYTA) IN A
CARIBBEAN REEF LAGOON.'' \emph{Journal of Phycology} 43 (1). John Wiley
\& Sons, Ltd: 69--77.
\url{https://doi.org/10.1111/j.1529-8817.2006.00307.x}.

\leavevmode\hypertarget{ref-Wefer1980}{}%
Wefer, Gerold. 1980. ``Carbonate production by algae Halimeda,
Penicillus and Padina.'' Vol. 285.
\url{https://www.nature.com/articles/285323a0.pdf?origin=ppub}.

\leavevmode\hypertarget{ref-Zieman1989}{}%
Zieman, Joseph C, James W Fourqurean, and Richard L Iverson. 1989.
``DISTRIBUTION, ABUNDANCE AND PRODUCTIVITY OF SEAGRASSES AND MACROALGAE
IN FLORIDA BAY.'' 1. Vol. 44.
\href{https://www.ingentaconnect.com/content/umrsmas/bullmar/1989/00000044/00000001/art00024?crawler=true\%7B/\&\%7Dcasa\%7B/_\%7Dtoken=opOrRAUO-uMAAAAA:uwe8u30Jk47-Xx2jqbvlwsNPp48AGfzmUl1DPqc9e5xyc4LQc2D7Tl2Bg0FeI70-brvnuU3KofL2yiGP\%7B/_\%7Dg}{https://www.ingentaconnect.com/content/umrsmas/bullmar/1989/00000044/00000001/art00024?crawler=true\{\textbackslash{}\&\}casa\{\textbackslash{}\_\}token=opOrRAUO-uMAAAAA:uwe8u30Jk47-Xx2jqbvlwsNPp48AGfzmUl1DPqc9e5xyc4LQc2D7Tl2Bg0FeI70-brvnuU3KofL2yiGP\{\textbackslash{}\_\}g}.

\end{document}
