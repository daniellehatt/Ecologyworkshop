\PassOptionsToPackage{unicode=true}{hyperref} % options for packages loaded elsewhere
\PassOptionsToPackage{hyphens}{url}
%
\documentclass[]{article}
\usepackage{lmodern}
\usepackage{amssymb,amsmath}
\usepackage{ifxetex,ifluatex}
\usepackage{fixltx2e} % provides \textsubscript
\ifnum 0\ifxetex 1\fi\ifluatex 1\fi=0 % if pdftex
  \usepackage[T1]{fontenc}
  \usepackage[utf8]{inputenc}
  \usepackage{textcomp} % provides euro and other symbols
\else % if luatex or xelatex
  \usepackage{unicode-math}
  \defaultfontfeatures{Ligatures=TeX,Scale=MatchLowercase}
\fi
% use upquote if available, for straight quotes in verbatim environments
\IfFileExists{upquote.sty}{\usepackage{upquote}}{}
% use microtype if available
\IfFileExists{microtype.sty}{%
\usepackage[]{microtype}
\UseMicrotypeSet[protrusion]{basicmath} % disable protrusion for tt fonts
}{}
\IfFileExists{parskip.sty}{%
\usepackage{parskip}
}{% else
\setlength{\parindent}{0pt}
\setlength{\parskip}{6pt plus 2pt minus 1pt}
}
\usepackage{hyperref}
\hypersetup{
            pdftitle={Ecology Workshop: Project Proposal},
            pdfauthor={Danielle Hatt},
            pdfborder={0 0 0},
            breaklinks=true}
\urlstyle{same}  % don't use monospace font for urls
\usepackage[margin=1in]{geometry}
\usepackage{graphicx,grffile}
\makeatletter
\def\maxwidth{\ifdim\Gin@nat@width>\linewidth\linewidth\else\Gin@nat@width\fi}
\def\maxheight{\ifdim\Gin@nat@height>\textheight\textheight\else\Gin@nat@height\fi}
\makeatother
% Scale images if necessary, so that they will not overflow the page
% margins by default, and it is still possible to overwrite the defaults
% using explicit options in \includegraphics[width, height, ...]{}
\setkeys{Gin}{width=\maxwidth,height=\maxheight,keepaspectratio}
\setlength{\emergencystretch}{3em}  % prevent overfull lines
\providecommand{\tightlist}{%
  \setlength{\itemsep}{0pt}\setlength{\parskip}{0pt}}
\setcounter{secnumdepth}{0}
% Redefines (sub)paragraphs to behave more like sections
\ifx\paragraph\undefined\else
\let\oldparagraph\paragraph
\renewcommand{\paragraph}[1]{\oldparagraph{#1}\mbox{}}
\fi
\ifx\subparagraph\undefined\else
\let\oldsubparagraph\subparagraph
\renewcommand{\subparagraph}[1]{\oldsubparagraph{#1}\mbox{}}
\fi

% set default figure placement to htbp
\makeatletter
\def\fps@figure{htbp}
\makeatother


\title{Ecology Workshop: Project Proposal}
\author{Danielle Hatt}
\date{1/10/2020}

\begin{document}
\maketitle

\hypertarget{research-statement}{%
\subsection{Research Statement}\label{research-statement}}

My research is in conjuction with Dr.~Ligia Collado-Vides based on
baseline data for standing stock, productivity and nutrients of green
calcareous macroalgae obtained as a participant of the Florida Coastal
Everglades long-term Ecological Research (FCE:LTER) program. With this
dataset, we are focused on biomass and nutrient tissue content of genera
\emph{Halimeda} and \emph{Penicillus} within three sites (Bob Allen
Keys, Duck Key and Sprigger Bank) over the past 10 years (2008-2018).

\hypertarget{objectives}{%
\subsection{Objectives}\label{objectives}}

Within Florida Bay, there is a gradient of higher nitrogen (N) and lower
phosphorus (P) from the northeast region of the bay to the southwest
regions of the bay as previously reported in seagrass studies
(Fourqurean et al.~1992, Frankovich and Fourqurean 1997). However, to my
knowledge, no studies show a nutrient gradient for calcareous macroalgae
in Florida Bay. My goal is to use a time series analysis to determine
and compare the fluctuation of standing stock and nutrient tissue
content between each genera and among the three sites throughout the
surveyed period without seasonality flcutuations as noise.

\hypertarget{hypotheses}{%
\subsection{Hypotheses}\label{hypotheses}}

\begin{enumerate}
\def\labelenumi{(\arabic{enumi})}
\tightlist
\item
  The standing stock will be genera dependent with expected larger
  contribution by Halimeda spp. than in \emph{Penicillus} spp.
\item
  The standing stock will be site dependent with a larger contribution
  expected towards to southwest region of the bay (SB) than the sites
  closer to the northeast region of the bay (BA, DK).
\item
  There will be no variability expected in nutrient tissue content
  between Halimeda and \emph{Penicillus} because they are both
  coenocytic and calcifying green macroalgae.
\item
  \emph{Penicillus} will show a spatial variability of nitrogen and
  phosphorus limitation with higher nitrogen and lower phosphorus
  towards the northeast region of the bay.
\end{enumerate}

\hypertarget{dataset}{%
\subsection{Dataset}\label{dataset}}

The dataset being used is public access data published as part of the
LTER:FCE program. Samples were collected by Seagrass Lab at Florida
International University and processed by the Marine Macroalgae Research
Lab (MMRL) at Florida International university.

\hypertarget{statistical-analysis}{%
\subsection{Statistical Analysis}\label{statistical-analysis}}

Statistical analyses were carried out using IBM SPSS software.
Parametric tests (one-way ANOVA) were used for normally distributed
sampling. Non-parametric tests (Mann-Whitney and Kruskal-Wallis) were
used for samples not normally distributed.

\end{document}
